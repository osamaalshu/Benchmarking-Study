\documentclass[11pt,a4paper]{article}

% Include all required packages
% Required packages for error analysis tables and figures
\usepackage{booktabs}        % Professional table formatting
\usepackage{array}           % Enhanced table column formatting
\usepackage{longtable}       % Tables spanning multiple pages
\usepackage{threeparttable}  % Table notes
\usepackage{multirow}        % Multi-row cells
\usepackage{graphicx}        % Include graphics
\usepackage{float}           % Better float positioning
\usepackage{caption}         % Enhanced captions
\usepackage{subcaption}      % Subfigures and subcaptions
\usepackage{pgfplots}        % For creating charts with TikZ
\usepackage{tikz}            % Drawing graphics
\usepackage{xcolor}          % Colors
\usepackage{amsmath}         % Mathematical formatting
\usepackage{siunitx}         % SI units and number formatting

% PGF plots settings
\pgfplotsset{compat=1.18}

% Color definitions for consistency
\definecolor{maunetcolor}{RGB}{31, 119, 180}      % Blue for MAUNet variants
\definecolor{traditionalcolor}{RGB}{255, 127, 14} % Orange for traditional models
\definecolor{errorcolor}{RGB}{214, 39, 40}        % Red for errors
\definecolor{successcolor}{RGB}{44, 160, 44}      % Green for success

% Table formatting commands
\newcommand{\bestvalue}[1]{\textbf{#1}}          % Bold for best values
\newcommand{\modelname}[1]{\textbf{#1}}          % Bold for model names

% Figure path (adjust according to your document structure)
\graphicspath{{error_analysis/results/}}

% Custom table notes environment
\newenvironment{tablenotes}
  {\begin{minipage}{\textwidth}\footnotesize}
  {\end{minipage}}

% Caption formatting
\captionsetup{
  font=small,
  labelfont=bf,
  format=plain,
  justification=justified,
  singlelinecheck=false
}


\title{Instance Segmentation Error Analysis\\Comprehensive Performance Evaluation}
\author{Your Name}
\date{\today}

\begin{document}

\maketitle

\section{Introduction}
This document presents a comprehensive error analysis of seven deep learning models for instance segmentation of cell images. The analysis covers 100 test images with detailed evaluation of error types, performance metrics, and model comparisons using micro-aggregated metrics and Hungarian matching for optimal assignment.

\section{Dataset and Methodology}
% Include dataset overview table
\begin{table}[htbp]
\centering
\caption{Error Analysis Dataset Overview}
\label{tab:dataset_overview}
\begin{tabular}{lr}
\toprule
\textbf{Metric} & \textbf{Value} \\
\midrule
Total Images Analyzed & 100 \\
Models Evaluated & 7 \\
Total Evaluations & 700 \\
Average GT Cells/Image & 454.4 \\
IoU Threshold & 0.5 \\
Minimum Instance Size & 10 pixels \\
\midrule
\multicolumn{2}{l}{\textbf{Models Analyzed:}} \\
\quad U-Net & Classic encoder-decoder architecture \\
\quad nnU-Net & Self-configuring U-Net variant \\
\quad SAC & Spatial Attention Convolution \\
\quad LSTM-UNet & LSTM-enhanced U-Net \\
\quad MAUNet-ResNet50 & Multi-scale attention with ResNet50 \\
\quad MAUNet-Wide & Wide multi-scale attention network \\
\quad MAUNet-Ensemble & Ensemble of MAUNet variants \\
\bottomrule
\end{tabular}
\begin{tablenotes}
\small
\item Note: Analysis excludes cell\_00101 due to data quality issues. All metrics computed using Hungarian matching for optimal assignment.
\end{tablenotes}
\end{table}


The error analysis methodology employs:
\begin{itemize}
\item \textbf{Hungarian Matching}: Optimal assignment algorithm for true positive determination
\item \textbf{Micro-Aggregation}: Sum TP/FP/FN across all images before computing precision/recall/F1
\item \textbf{IoU Threshold}: 0.5 for true positive classification
\item \textbf{Enhanced Visualization}: Automatic contrast enhancement for challenging images
\item \textbf{Comprehensive Error Categorization}: FN, FP, splits, merges with detailed analysis
\end{itemize}

\section{Model Performance Results}

\subsection{Overall Performance Comparison}
% Include main performance table
\begin{table}[htbp]
\centering
\caption{Model Performance Summary - Micro-Aggregated Metrics}
\label{tab:model_performance}
\begin{tabular}{lcccccc}
\toprule
\textbf{Model} & \textbf{F1-Score} & \textbf{PQ} & \textbf{Precision} & \textbf{Recall} & \textbf{RQ} & \textbf{SQ} \\
\midrule
\textbf{MAUNet-Wide} & \textbf{0.529} & 0.413 & 0.605 & 0.470 & 0.529 & 0.781 \\
\textbf{MAUNet-ResNet50} & \textbf{0.507} & 0.399 & 0.598 & 0.440 & 0.507 & 0.787 \\
\textbf{MAUNet-Ensemble} & \textbf{0.499} & \textbf{0.437} & \textbf{0.632} & 0.413 & 0.499 & \textbf{0.876} \\
nnU-Net & 0.357 & 0.278 & 0.367 & 0.348 & 0.357 & 0.778 \\
U-Net & 0.315 & 0.239 & 0.335 & 0.297 & 0.315 & 0.758 \\
LSTM-UNet & 0.282 & 0.199 & 0.263 & 0.305 & 0.282 & 0.706 \\
SAC & 0.003 & 0.002 & 0.005 & 0.002 & 0.003 & 0.667 \\
\bottomrule
\end{tabular}
\begin{tablenotes}
\small
\item Note: Bold values indicate best performance in each metric. Metrics calculated using micro-aggregation across 100 test images. PQ = Panoptic Quality, RQ = Recognition Quality, SQ = Segmentation Quality.
\end{tablenotes}
\end{table}


% Include performance comparison chart
% Performance comparison chart - can be generated with pgfplots or included as image
\begin{figure}[htbp]
\centering
\begin{tikzpicture}
\begin{axis}[
    ybar,
    width=\textwidth,
    height=8cm,
    xlabel={Model},
    ylabel={Performance Score},
    symbolic x coords={SAC, LSTM-UNet, U-Net, nnU-Net, MAUNet-ResNet50, MAUNet-Wide, MAUNet-Ensemble},
    xtick=data,
    x tick label style={rotate=45, anchor=east},
    legend style={at={(0.5,-0.2)}, anchor=north, legend columns=3},
    ymin=0,
    ymax=1.0
]

% F1-Score bars
\addplot coordinates {
    (SAC, 0.003)
    (LSTM-UNet, 0.282)
    (U-Net, 0.315)
    (nnU-Net, 0.357)
    (MAUNet-ResNet50, 0.507)
    (MAUNet-Wide, 0.529)
    (MAUNet-Ensemble, 0.499)
};

% PQ bars
\addplot coordinates {
    (SAC, 0.002)
    (LSTM-UNet, 0.199)
    (U-Net, 0.239)
    (nnU-Net, 0.278)
    (MAUNet-ResNet50, 0.399)
    (MAUNet-Wide, 0.413)
    (MAUNet-Ensemble, 0.437)
};

% Precision bars
\addplot coordinates {
    (SAC, 0.005)
    (LSTM-UNet, 0.263)
    (U-Net, 0.335)
    (nnU-Net, 0.367)
    (MAUNet-ResNet50, 0.598)
    (MAUNet-Wide, 0.605)
    (MAUNet-Ensemble, 0.632)
};

\legend{F1-Score, PQ, Precision}
\end{axis}
\end{tikzpicture}
\caption{Model Performance Comparison Across Key Metrics}
\label{fig:performance_comparison}
\begin{quote}
\small
Comparative bar chart showing F1-Score, Panoptic Quality (PQ), and Precision across all evaluated models. MAUNet architectures demonstrate superior performance, with MAUNet-Ensemble achieving highest precision (0.632) and PQ (0.437), while MAUNet-Wide achieves best F1-Score (0.529). Traditional architectures (U-Net, nnU-Net) show moderate performance, while SAC exhibits poor performance across all metrics. The chart clearly illustrates the effectiveness of multi-scale attention mechanisms in cell instance segmentation tasks.
\end{quote}
\end{figure}


The results demonstrate clear performance hierarchies, with MAUNet architectures achieving superior performance across all metrics. MAUNet-Wide achieves the highest F1-score (0.529), while MAUNet-Ensemble excels in precision (0.632) and Panoptic Quality (0.437).

\subsection{Panoptic Quality Analysis}
% Include PQ breakdown table
\begin{table}[htbp]
\centering
\caption{Panoptic Quality Analysis - Detailed Breakdown}
\label{tab:panoptic_quality}
\begin{tabular}{lccc}
\toprule
\textbf{Model} & \textbf{PQ} & \textbf{RQ} & \textbf{SQ} \\
\midrule
\textbf{MAUNet-Ensemble} & \textbf{0.437} & 0.499 & \textbf{0.876} \\
\textbf{MAUNet-Wide} & 0.413 & \textbf{0.529} & 0.781 \\
\textbf{MAUNet-ResNet50} & 0.399 & 0.507 & 0.787 \\
nnU-Net & 0.278 & 0.357 & 0.778 \\
U-Net & 0.239 & 0.315 & 0.758 \\
LSTM-UNet & 0.199 & 0.282 & 0.706 \\
SAC & 0.002 & 0.003 & 0.667 \\
\bottomrule
\end{tabular}
\begin{tablenotes}
\small
\item Note: Bold values indicate best performance in each metric.
\item PQ (Panoptic Quality) = RQ × SQ, measures overall segmentation performance
\item RQ (Recognition Quality) = Detection performance (similar to F1-score)
\item SQ (Segmentation Quality) = IoU quality of matched instances
\item MAUNet-Ensemble achieves highest PQ through superior segmentation quality
\item MAUNet-Wide achieves highest recognition quality (detection performance)
\end{tablenotes}
\end{table}


Panoptic Quality provides a comprehensive measure combining both detection and segmentation performance. The decomposition into Recognition Quality (RQ) and Segmentation Quality (SQ) reveals that MAUNet-Ensemble achieves superior segmentation quality, while MAUNet-Wide excels in detection performance.

\section{Error Analysis}

\subsection{Error Type Distribution}
% Include error analysis table
\begin{table}[htbp]
\centering
\caption{Error Analysis Summary - Average Error Counts Per Image}
\label{tab:error_analysis}
\begin{tabular}{lccccc}
\toprule
\textbf{Model} & \textbf{False Negatives} & \textbf{False Positives} & \textbf{Splits} & \textbf{Merges} & \textbf{Total Errors} \\
\midrule
\textbf{MAUNet-Ensemble} & \textbf{241.0} & \textbf{109.0} & \textbf{1.3} & \textbf{28.3} & \textbf{379.6} \\
\textbf{MAUNet-Wide} & \textbf{241.0} & 139.1 & \textbf{1.3} & 35.2 & 416.6 \\
\textbf{MAUNet-ResNet50} & 254.7 & 134.6 & \textbf{0.9} & 36.9 & 427.1 \\
nnU-Net & 296.4 & 272.7 & 3.6 & 47.3 & 620.0 \\
U-Net & 319.6 & 268.0 & 3.1 & 53.1 & 643.8 \\
LSTM-UNet & 315.8 & 389.0 & 1.7 & 41.8 & 748.3 \\
SAC & 453.5 & 179.0 & 0.1 & 8.2 & 640.8 \\
\bottomrule
\end{tabular}
\begin{tablenotes}
\small
\item Note: Bold values indicate best (lowest) performance in each error category. Analysis based on 100 test images with average 454.4 ground truth cells per image.
\item False Negatives: Missed cells that should have been detected
\item False Positives: Incorrect detections (artifacts segmented as cells)
\item Splits: Ground truth cells divided into multiple predictions (over-segmentation)
\item Merges: Multiple ground truth cells combined into single prediction (under-segmentation)
\end{tablenotes}
\end{table}


% Include error distribution chart
% Error distribution chart
\begin{figure}[htbp]
\centering
\begin{tikzpicture}
\begin{axis}[
    ybar stacked,
    width=\textwidth,
    height=8cm,
    xlabel={Model},
    ylabel={Average Error Count per Image},
    symbolic x coords={MAUNet-Ensemble, MAUNet-Wide, MAUNet-ResNet50, nnU-Net, U-Net, LSTM-UNet, SAC},
    xtick=data,
    x tick label style={rotate=45, anchor=east},
    legend style={at={(0.5,-0.2)}, anchor=north, legend columns=2},
    ymin=0,
    ymax=800
]

% False Negatives (bottom layer)
\addplot coordinates {
    (MAUNet-Ensemble, 241.0)
    (MAUNet-Wide, 241.0)
    (MAUNet-ResNet50, 254.7)
    (nnU-Net, 296.4)
    (U-Net, 319.6)
    (LSTM-UNet, 315.8)
    (SAC, 453.5)
};

% False Positives
\addplot coordinates {
    (MAUNet-Ensemble, 109.0)
    (MAUNet-Wide, 139.1)
    (MAUNet-ResNet50, 134.6)
    (nnU-Net, 272.7)
    (U-Net, 268.0)
    (LSTM-UNet, 389.0)
    (SAC, 179.0)
};

% Splits
\addplot coordinates {
    (MAUNet-Ensemble, 1.3)
    (MAUNet-Wide, 1.3)
    (MAUNet-ResNet50, 0.9)
    (nnU-Net, 3.6)
    (U-Net, 3.1)
    (LSTM-UNet, 1.7)
    (SAC, 0.1)
};

% Merges
\addplot coordinates {
    (MAUNet-Ensemble, 28.3)
    (MAUNet-Wide, 35.2)
    (MAUNet-ResNet50, 36.9)
    (nnU-Net, 47.3)
    (U-Net, 53.1)
    (LSTM-UNet, 41.8)
    (SAC, 8.2)
};

\legend{False Negatives, False Positives, Splits, Merges}
\end{axis}
\end{tikzpicture}
\caption{Error Type Distribution Across Models}
\label{fig:error_distribution}
\begin{quote}
\small
Stacked bar chart showing the distribution of different error types across all evaluated models. Models are ordered by total error count (lowest to highest). Key observations: (1) False negatives dominate error patterns across all models, (2) MAUNet variants show significantly lower total errors, (3) LSTM-UNet suffers from high false positive rates, (4) Split and merge errors are relatively rare across all architectures. This visualization guides optimization efforts by highlighting the primary error sources for each model architecture.
\end{quote}
\end{figure}


The error analysis reveals that false negatives (missed cells) constitute the primary error source across all models, followed by false positives. Split and merge errors are relatively rare, indicating that most models handle cell boundary detection adequately when cells are detected.

\subsection{Key Findings}
\begin{itemize}
\item \textbf{False Negatives Dominate}: All models struggle primarily with missing cells rather than incorrect segmentation
\item \textbf{MAUNet Superiority}: Multi-scale attention mechanisms significantly reduce all error types
\item \textbf{Precision vs Recall Trade-off}: MAUNet-Ensemble optimizes precision while MAUNet-Wide balances precision and recall
\item \textbf{Traditional Model Limitations}: U-Net and nnU-Net show moderate performance with higher error rates
\end{itemize}

\section{Visual Analysis}

\subsection{Comprehensive Error Visualization}
% Include comprehensive visualization figures
\begin{figure}[htbp]
\centering
\includegraphics[width=\textwidth]{results/error_analysis/comprehensive_visualizations/cell_00070_comprehensive_analysis.png}
\caption{Comprehensive Error Analysis Example - Cell Image 00070}
\label{fig:comprehensive_analysis_example}
\begin{quote}
\small
Comprehensive 4×7 visualization showing: (Column 1) Enhanced original image with improved contrast for challenging low-contrast images; (Column 2) Ground truth instance segmentation with 454 average cells per image; (Column 3) Model predictions for all 7 architectures with cell counts; (Column 4) Error overlay visualization with color-coded analysis: red = false negatives (missed cells), blue = false positives (incorrect detections), green = true positives (correct matches), black = background. Each row represents a different model, demonstrating varying performance across architectures. MAUNet variants (rows 5-7) show superior performance with fewer errors compared to traditional approaches.
\end{quote}
\end{figure}

% Additional example for challenging cases
\begin{figure}[htbp]
\centering
\includegraphics[width=\textwidth]{results/error_analysis/comprehensive_visualizations/cell_00073_comprehensive_analysis.png}
\caption{Challenging Case Analysis - Cell Image 00073}
\label{fig:challenging_case_analysis}
\begin{quote}
\small
Example of challenging low-contrast image analysis showing the effectiveness of image enhancement techniques and model robustness. Originally appearing as blank/white image, enhanced visualization reveals cellular structures enabling proper error analysis. Note the significant performance differences between models on difficult cases, with MAUNet architectures maintaining better detection capabilities compared to classical U-Net approaches.
\end{quote}
\end{figure}


\subsection{Success and Failure Case Analysis}
% Include success/failure analysis figures
\begin{figure}[htbp]
\centering
\includegraphics[width=\textwidth]{results/error_analysis/new_visualizations/maunet_ensemble_success_failure_analysis.png}
\caption{MAUNet-Ensemble Success and Failure Case Analysis}
\label{fig:maunet_ensemble_analysis}
\begin{quote}
\small
Model-specific analysis showing best (left) and worst (right) performing cases for MAUNet-Ensemble. Success cases demonstrate optimal conditions where the model achieves high precision and recall: clear cell boundaries, moderate density, good contrast. Failure cases reveal challenging scenarios: overlapping cells, poor contrast, high density regions, and irregular morphologies. This analysis provides insights into model limitations and potential improvement directions.
\end{quote}
\end{figure}

\begin{figure}[htbp]
\centering
\includegraphics[width=\textwidth]{results/error_analysis/new_visualizations/maunet_wide_success_failure_analysis.png}
\caption{MAUNet-Wide Success and Failure Case Analysis}
\label{fig:maunet_wide_analysis}
\begin{quote}
\small
Success and failure analysis for MAUNet-Wide, the best overall F1-score performer. The model shows robust performance across diverse conditions but struggles with extremely dense regions and poor contrast scenarios. Comparison with MAUNet-Ensemble reveals complementary strengths: MAUNet-Wide excels in detection (higher recall) while MAUNet-Ensemble excels in precision (fewer false positives).
\end{quote}
\end{figure}


\subsection{Dataset Difficulty Analysis}
% Include best/worst analysis figure
\begin{figure}[htbp]
\centering
\includegraphics[width=\textwidth]{results/error_analysis/new_visualizations/best_worst_images_overall.png}
\caption{Best vs Worst Performing Images Analysis}
\label{fig:best_worst_analysis}
\begin{quote}
\small
Comparative analysis showing the 5 most challenging (top row) and 5 easiest (bottom row) images based on average F1-scores across all models. Each image pair shows the enhanced original image and corresponding ground truth annotations. Challenging images typically exhibit: (1) low contrast requiring significant enhancement, (2) high cell density with overlapping boundaries, (3) irregular cell shapes and sizes. Easy images demonstrate: (1) clear contrast and well-defined boundaries, (2) moderate cell density with good separation, (3) consistent cell morphology. This analysis helps identify dataset characteristics that impact model performance and guides future data collection strategies.
\end{quote}
\end{figure}


\section{Conclusions}

The comprehensive error analysis demonstrates:

\begin{enumerate}
\item \textbf{Architecture Impact}: Multi-scale attention mechanisms (MAUNet variants) significantly outperform traditional encoder-decoder architectures
\item \textbf{Error Patterns}: False negatives represent the primary challenge, suggesting need for improved feature extraction and detection sensitivity
\item \textbf{Model Complementarity}: Different MAUNet variants excel in different aspects (precision vs recall), suggesting potential for ensemble approaches
\item \textbf{Dataset Challenges}: Image quality and cell density significantly impact performance across all models
\end{enumerate}

\section{Recommendations}

Based on the analysis:
\begin{itemize}
\item \textbf{Production Deployment}: Use MAUNet-Wide for best overall performance
\item \textbf{High-Precision Applications}: Deploy MAUNet-Ensemble when false positives are critical
\item \textbf{Future Research}: Focus on reducing false negatives through improved feature extraction
\item \textbf{Data Collection}: Prioritize high-quality, well-contrasted images for optimal performance
\end{itemize}

\end{document}
