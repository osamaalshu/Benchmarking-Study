\begin{table}[htbp]
\centering
\caption{Panoptic Quality Analysis - Detailed Breakdown}
\label{tab:panoptic_quality}
\begin{tabular}{lccc}
\toprule
\textbf{Model} & \textbf{PQ} & \textbf{RQ} & \textbf{SQ} \\
\midrule
\textbf{MAUNet-Ensemble} & \textbf{0.437} & 0.499 & \textbf{0.876} \\
\textbf{MAUNet-Wide} & 0.413 & \textbf{0.529} & 0.781 \\
\textbf{MAUNet-ResNet50} & 0.399 & 0.507 & 0.787 \\
nnU-Net & 0.278 & 0.357 & 0.778 \\
U-Net & 0.239 & 0.315 & 0.758 \\
LSTM-UNet & 0.199 & 0.282 & 0.706 \\
SAC & 0.002 & 0.003 & 0.667 \\
\bottomrule
\end{tabular}
\begin{tablenotes}
\small
\item Note: Bold values indicate best performance in each metric.
\item PQ (Panoptic Quality) = RQ × SQ, measures overall segmentation performance
\item RQ (Recognition Quality) = Detection performance (similar to F1-score)
\item SQ (Segmentation Quality) = IoU quality of matched instances
\item MAUNet-Ensemble achieves highest PQ through superior segmentation quality
\item MAUNet-Wide achieves highest recognition quality (detection performance)
\end{tablenotes}
\end{table}
