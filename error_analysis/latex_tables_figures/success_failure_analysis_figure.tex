\begin{figure}[htbp]
\centering
\includegraphics[width=\textwidth]{results/error_analysis/new_visualizations/maunet_ensemble_success_failure_analysis.png}
\caption{MAUNet-Ensemble Success and Failure Case Analysis}
\label{fig:maunet_ensemble_analysis}
\begin{quote}
\small
Model-specific analysis showing best (left) and worst (right) performing cases for MAUNet-Ensemble. Success cases demonstrate optimal conditions where the model achieves high precision and recall: clear cell boundaries, moderate density, good contrast. Failure cases reveal challenging scenarios: overlapping cells, poor contrast, high density regions, and irregular morphologies. This analysis provides insights into model limitations and potential improvement directions.
\end{quote}
\end{figure}

\begin{figure}[htbp]
\centering
\includegraphics[width=\textwidth]{results/error_analysis/new_visualizations/maunet_wide_success_failure_analysis.png}
\caption{MAUNet-Wide Success and Failure Case Analysis}
\label{fig:maunet_wide_analysis}
\begin{quote}
\small
Success and failure analysis for MAUNet-Wide, the best overall F1-score performer. The model shows robust performance across diverse conditions but struggles with extremely dense regions and poor contrast scenarios. Comparison with MAUNet-Ensemble reveals complementary strengths: MAUNet-Wide excels in detection (higher recall) while MAUNet-Ensemble excels in precision (fewer false positives).
\end{quote}
\end{figure}
